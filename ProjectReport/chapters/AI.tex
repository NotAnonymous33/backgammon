\chapter{AI Algorithms and Architectures}
Before describing what techniques can be used to implement an AI system which plays backgammon, it is important to describe the game itself and the different types of AI systems that can be used to play it.
Backgammon is a two player, zero sum game with chance and perfect information.
Zero sum: One player's advantage is equivalent to the other player's loss. The sum of the players scores is always zero.
Perfect information: Both players have complete knowledge of the game state at all times. There is no hidden information.
Chance: The outcome of the game is not deterministic. The moves a player can make, and the outcome of the game depends on the rolls of the dice.

For such games, the best systems (such as AlphaGo) used a trained policy network, combined with Monte Carlo Tree Search (MCTS) in real-time gameplay \cite{aiparadigms}.
MCTS is limited to games with perfect information, and also requires that the action space is discrete, which are both true for backgammon.

This project includes different types of AI systems that can be used to play backgammon, such as rule-based systems, neural networks, and MCTS.

\section{Heuristics}
Heuristic approaches involve estimating the desirability of a given board state using rules which have been shown to work through trial and improvement. These rules typically combine several features based on expert knowledge, such as pip count difference (a measure of race progress), number of blots, strength of blockades (primes), and control of key points (anchors) [3, 5]. The agent then selects the move leading to the state with the best heuristic evaluation. While intuitive, designing effective heuristics can be complex, and heuristic rules may struggle with the nuances and dynamic nature of the game.

[3] Russell, S. J., \& Norvig, P. (2020). Artificial Intelligence: A Modern Approach (4th ed.). Prentice Hall. (Standard AI textbook covering game playing, search, RL).
[4] Campbell, M., Hoane, A. J., \& Hsu, F. H. (2002). Deep Blue. Artificial Intelligence, 134(1-2), 57-83. (Example of classical approach in chess).


\section{Stochastic Gradient Descent (SGD)}
While heuristic evaluation functions provide a computationally cheap way to estimate the strength of a board position, manually tuning the weights associated with different features can be challenging, subjective, and may lead to suboptimal performance. SGD offers a principled and automated method for learning these weights based on game experience.

The heuristic function $H(s, \boldsymbol{w})$ calculates the value of state $s$ as a linear combination of features $f_i(s)$ with corresponding weights $w_i$. The goal of SGD is to find the optimal weights $\boldsymbol{w}$ such that $H(s, \boldsymbol{w})$ gives the most accurate estimate for the value of being in a given state.

In the context of self-play, a common target is the final outcome $T$ of a game (e.g. +1 for a win and -1 for a loss). A loss function which measures the difference between the heuristic prediction and the target outcome can then be defined. A common choice for a loss function is the mean squared error (MSE):
$$
L(\boldsymbol{w}) = E[(T - H(s, \boldsymbol{w})^2]
$$
where $E[...]$ denotes the expectation over all possible states during the game.

The goal of SGD can be redefined as minimising the loss function $L(\boldsymbol{w})$.

Gradient descent algorithms work by iteratively updating the values of $\boldsymbol{w}$ by moving the values of $\boldsymbol{w}$ in the opposite direction of the gradient. This results in the following update rule:
$$
\boldsymbol{w} \leftarrow \boldsymbol{w} - \eta \nabla L(\boldsymbol{w})
$$
where $\eta$ is the learning rate, a value which determines the size of the steps the update rule makes.

The problem with this is calculating the gradient of the loss function requires summing over the entire distribution of states, which is computationally intractable. SGD overcomes this limitation by approximating the true gradient by estimating the gradient using only a few training examples at each step. This has the added benefit of helping to escape local maxima. TODO cite that paper in the other tab.

Over many iterations, these adjustments will move the weights $\boldsymbol{w}$ towards the value that minimises the prediction error, i.e. the heuristic function is as accurate as possible in predicting the value of a given state. [12] Bottou, L. (2010). Large-scale machine learning with stochastic gradient descent. Proceedings of COMPSTAT'2010, 177-186. (Provides a good overview of SGD for large-scale learning).


\section{Monte Carlo Tree Search}
Monte Carlo methods are a type of algorithm which estimates an unknown value based on random sampling. They are often used to solve problems that are deterministic in principle but difficult or impossible to solve with exact analytical methods TODO cite Russell and norvig. By generating many random samples and observing the outcomes, Monte Carlo methods utilise the Law of Large Numbers to approximate solutions.

TODO fun fact
The Law of Large Numbers states that the sample mean of $n$ independent and identically distributed random variables with mean $\mu$ approaches $\mu$ as $n$ tends to infinity.

Monte Carlo Tree Search (MCTS) is a Monte Carlo method which is used to search game trees to find optimal moves. Because of its stochastic nature, it is especially well suited to games which include chance, such as backgammon TODO cite [Browne, C. B., Powley, E., Whitehouse, D., Lucas, S. M., Cowling, P. I., Rohlfshagen, P., … \& Colton, S. (2012). A survey of Monte Carlo tree search methods. \_IEEE Transactions on Computational Intelligence and AI in Games, 4\_(1), 1-43.] 
TODO talk about minimax and how mcts approaches minimax

TODO cite https://link.springer.com/article/10.1007/s10462-022-10228-y
TODO talk about game trees in the AI and Games section as a definition. 
Intuition: A game tree is created with the current state as the root node. From then, random moves are made by all players until the end of the game (a leaf node) is reached, (or some other critieria is met such as TODO). The results of these games are propagated back throughout the tree and the value of each node is updated. By doing this a large number of times, the expected value of each node can be estimated, and the move resulting in the highest expectation can be picked.

The algorithm iteratively runs 4 steps until termination. 

\begin{enumerate}
    \item \textbf{Selection}: Starting at the root node, a child node is chosen according to a tree policy. This policy tries to balance exploring unvisited nodes (which may have high potential), with exploiting nodes which have been shown to be promising.
    \item \textbf{Expansion}: If the selected child node does not represent a terminal game state, it is then added to the stored game tree. 
    \item \textbf{Simulation}: From this newly added node, a simulation is run by randomly choosing moves for both players. This is continued until the end of the game is reached, or until another condition is met, such as reaching a move limit, or reaching a state where calculating the winner can be done easily.
    \item \textbf{Backpropagation}: The outcome of the game is backpropagated throughout the tree, from the expanded node to the root node. The statistics of each node along the path is updated, for example, the expected value of the node, and how many times the node has been visited. 
\end{enumerate}

This cycle is repeated until some condition is met. This could be until a certain amount of time has elapsed, or until some number of simulations has been reached. Finally, a move is chosen from the children of the root node, typically this is the node with the highest estimated value or the highest visit count.

There are many enhancements that can increase the performance of MCTS, many of which are described in "A Survey of Monte Carlo Tree Search Methods" \cite{Browne2012}. Some enhancements which are more applicable to backgammon are listed below.

\begin{itemize}
    \item One obvious improvement is to reuse trees that have already been computed previously. When an updated board has been passed to the algorithm, the grandchildren (because child nodes correspond to the opponent's turn; one more move something TODO to the player) can be checked. This will allow subsequent moves to have more accurate statistics on their child nodes, reducing wasted computation and allowing better moves to be chosen. Due to backgammon's large branching factor, it is important that the MCTS implementation can complete as many iterations as possible TODO too vague fix. If this is not the case, the likelihood of the grandchild node being computed will be low (or will have low N TODO). Due to the time taken searching the game tree for the grandchild, the agent will have less time to do further rollouts, which will result in worse performance.  

    TODO: A possible speed improvement was tested where the board was hashed and compared with the grandchildren. By creating a new tree rather than blah blah blah, improvement. however in practice, the branching factor resulted in there being very few common grandchildren. With the hashing and searching, this decreased the performance of the agent (where the agent was given 1 second to search). If the agent was given longer to search (thus resulting in a larger expanded game tree), it is possible that the performance would increase, as it would become increasingly more likely that a child had been explored.
    
    \item UCB1-Tuned \cite{6145622}: TODO cite  P. Auer, N. Cesa-Bianchi, and P. Fischer, “Finite-time analysis of the
    multiarmed bandit problem,” Mach. Learn., vol. 47, no. 2, pp. 235–256, 2002. One of the main problems in implementing Monte Carlo Tree Search is choosing the tree policy for selection. The tree policy needs to balance the exploration-exploitation tradeoff. UCB1-tuned offers a tigher bound compared with the more commonly used UCB1 algorithm TODO cite [8] Kocsis, L., \& Szepesvári, C. (2006). Bandit based Monte-Carlo planning. Proceedings of the 17th European Conference on Machine Learning, 282-293. When choosing which child node to explore, UCB1-tuned chooses the child which maximises the value.
    $$
    \mathrm{UCB} 1 \text{-Tuned}=\bar{X}_j+2C\sqrt{\frac{\ln n}{n_j} \min \left\{\frac{1}{4}, V_j\left(n_j\right)\right\}}
    $$ 
    where 
    $$
    V_j(s)=\left(1 / 2 \sum_{\tau=1}^s X_{j, \tau}^2\right)-\bar{X}_{j, s}^2+\sqrt{\frac{2 \ln t}{s}}
    $$ in the situation that child node $j$ has been chosen $s$ times in the first $t$ simulations, $\bar{X_j}$ is the average reward of child node $j$, $n$ is the total number of simulations, $n_j$ is the number of times node $j$ has been chosen already, and $C$ is a constant exploration parameter (where a higher value of C results in more exploration). 
    
    The first term $\bar{X_j}$ represents "exploitation", where a higher expected value will result in the node being picked more often, while the other term represents the "exploration" term, where values which have been picked a lower proportion of the time have a higher value.

    \item Rapid Action Value Estimation (RAVE): RAVE is a popular method for improving the information stored in each nodes statistics, particularly in Go programs which leads to faster convergence during the early stages of search \cite{GellySilver2007}. For each move, statistics across all simulations where that move appears is updated, with the assumption that a move that performs well in one particular is likely to perform well in other similar states. The RAVE and UCB values are then combined to estimate a nodes value.

    \item Combining MCTS with other models \cite{44806} \cite{Browne2012}: 
    One of the biggest improvements in MCTS is the integration of value networks to evaluate board positions, and policy networks to select moves. 
    By using a policy network, nodes with less value can be quickly identified and do not need to be expanded.
    By using a value network, games to not need to be expanded to the end.
    Instead the simulation is run to a predetermined depth, at which point the value network is used to evaluate the board position.
    Since the value network will be provided boards which are closer to the end of the game, it will be able to provide an accurate estimate of the value of the board position.
    This allows for more iterations of MCTS, increasing performance. The most notable example of this is AlphaGo, which used a combination of MCTS, policy networks, and value networks to achieve superhuman performance in Go, achieving a 99.8\% winrate against other Go programs \cite{44806}.

    
TODO definition: exploration-exploitation problem and what it entails / how it is applied in mcts. 
\end{itemize} 

Ross, Benjamin, and Munson present a closed form approximation for the probability that a player wins a backgammon race \cite{estimating}.
By using the Raw Pip Count (RPC) of each player, a Single Checker Model (SCM) is created is used to approximates each player's position as a single checker which needs to travel a distance equal to the RPC.
An existing approximation based on the central limit theorem in renewal theory is then used to estimate the SCM probabilities. 
After comparing this approximation to simulations of actual backgammon races, the formula was adjusted to account for bearoff rules and pip wastage, which the original SCM approximation did not take into account. This leads to the following value:
$$
\frac{\Delta^2 + \Delta / 7}{S - 25}
$$
where $\Delta = Y - X + 4$ is the adjusted difference in RPC between the two players, and $S = X + Y$ is the sum of the two players RPCs, where $X$ is the lower pip count, and $Y$ is the higher pip count of the two players. 
This value could then be compared with a precomputed lookup table to determine an approximate probability of winning the race (and therefore the game). 
This model was originally designed to be used by players to estimate the probability of winning during the game using only mental arithmetic, and therefore has a number of limitations. The model does not take into account hitting or any other interations, such as when a player is blocked, and some simplifications to the formula were made so players could easily calculate values mentally. 


TODO: define renewal theory and central limit theorem

----------------------------------------------
\section{Neural Networks} 

A neural network is a computational model which maps inputs to outputs using a series of transformations.
In particular, multilayer perceptrons (MLPs) model consists of an input layer, one or more hidden layers, and an output layer, where each layer is made up of interconnected nodes (neurons) that calculate a weighted sum of its inputs, and applies a non-linear activation function to produce an output. 
The universal approximation theorem states that a large enough MLP can approximate any continuous function to arbitrarily high accuracy \cite{Hornik1989}.
In the context of backgammon, the input layer would represent a board state, and the output layer would represent the probability of winning from that state.
In a deep neural network, only the board state is passed to the first layer, and the higher level features would be learned by the network as it is trained, however this required a large amount of training data and computation. 
A simpler approach is to use a network with less neurons and less layers, thus needing less training data and computation, and passing in the features of the board state as inputs (e.g. number of blots, number of anchor points, etc. TODO check if ive defined these above).

\section{Reinforcement Learning}

Reinforcement learning is a type of machine learning where an agent learns by interacting with an environment to maximise reward, as opposed to learning from labelled inputs and outputs (supervised learning) or learning from unlabelled data (unsupervised learning). By taking actions in an environment and observing the outcome, the agent is able to choose actions which maximise the reward.  
One technique for training an agent using reinforcement learning is to implement self-play, where the agent plays against itself and learns from the outcomes. 
As backgammon naturally has chance, the agent is forced to explore a wide variety of boards, which helps to avoid the problem of converging to a local minimum. 
This is unlike with stochastic gradient descent for two reasons: first, the neural network learns from every board state, while SGD learns only from the outcome; second, the neural network aims to learn the value of a board state, while SGD aims to learn a strategy for playing the game i.e. the neural network needs to see different positions to learn (which is natural to backgammon), while SGD needs to play against different styles to learn. 

From this, an initial model could be created which learns in the following way:
\begin{enumerate}
    \item The agent plays a game against itself, and the final outcome is recorded.
    \item The neural network is trained on every board state it has seen in the game, using the final outcome as the target value.
    \item The agent continues to play against itself and learn from the outcomes until some point is reached.
\end{enumerate}

The problem with this approach, is that by assigning each board state the same value, the implicit assumption is that all moves are equally important, which is not the case and causes the network to overestimate the value of early game states.
For example, if the agent wins a game but makes many mistakes along the way, the agent may reinforce the mistakes it made, as even bad states would be given a positive reward. 

To overcome this, temporal difference learning (TD learning) was developed, which was described by Sutton and Barto as "one idea central and novel to reinforcement learning" \cite{Sutton2018}.
The way TD learning overcomes this problem is by updating values at each time step based on the expected future rewards (multiplied by some factor $\gamma$), rather than just the final outcome. This comes with two main benefits:
\begin{itemize}
    \item By updating at every step, the agent is able to identify good and bad moves, and assign rewards accordingly.
    \item The agent is able to learn continuously after every action, rather than waiting until the end of a game to learn.
\end{itemize}
The reason expected future rewards are decayed by $\gamma$ is because rewards become less valuble the longer a player has to wait for them. 
This is because with longer wait times, the more variance there is in the outcome, and it is more likely that there will be errors. 
By having a decay factor, the agent is encouraged to take shorter, more certain paths to the goal.

This raises another problem, which is that initially during learning, only the states which lead to a win or loss have an associated reward, meaning learning can be extremely slow.
TODO this wording is shocking: The idea of TD-$\lambda$ is to solve this problem by updating the value of every past state in a sequence that led to the reward.
A value is chosen for $\lambda$ between 0 and 1, which determines the decay of the reward, with higher values giving more weight to states which are further away.
When a reward $x$ is received to a state $s$ at time step $t$, the value of previous states $s_{t-1}, s_{t-2}, ...$ are given a reward of $x (\lambda \times \gamma)^{t - i}$, where $i$ is how many steps back the state was seen.

Famously, TD-$\lambda$ was used in TD-Gammon \cite{Tesauro1995}, which was the first backgammon program to achieve expert level play.
It worked by using an MLP to predict the outcome of a game from the given board position. 
When making a move, the program would generate all legal moves from the current position, and use the network to evaluate each possible position.
The move with the highest predicted value was then chosen.

During training, the network started from a random initial state, and played against itself over and over. 
As training progressed, basic strategies began to emerge such as hitting blots, and creating anchors. 
TD-Gammon 1.0 contained 198 input nodes and 40 hidden units, and was trained for 200,000 games, resulting in an agent that was able to win regional tournaments.
TD-Gammon 2.1 was close to the world's best players, and was able to discover new strategies which changed how humans play certain board positions.
Pollack and Blair \cite{Pollack1997} argue that TD-Gammon's success was due to learning through self play, by showing a simpler model which learned effectively through the same method. 

TODO in the introduction talk about how AI in games helps create new AI techniques, which can be applied to other fields.

“TD-Gammon has definitely come into its own. There is no question in my mind that its positional judgment is far better than mine.
Only on small technical areas can I claim a definite advantage over
it . . . . I find a comparison of TD-Gammon and the high-level chess
computers fascinating. The chess computers are tremendous in
tactical positions where variations can be calculated out. Their
weakness is in vague positional games, where it is not obvious
what is going on . . . . TD-Gammon is just the opposite. Its strength
is in the vague positional battles where judgment, not calculation ,
is the key. There, it has a definite edge over humans . . . . In particular, its judgment on bold vs. safe play decisions, which is what
backgammon really is all about, is nothing short of phenomenal . . . .
Instead of a dumb machine which can calculate things much faster
than humans such as the chess playing computers, you have built
a smart machine which learns from experience pretty much the
same way that humans do”, Kit Woolsey, a top 10 world backgammon player, quoted in \cite{Tesauro1995}.

TODO: describe momentum, learning rate decay, parameter specific learning rate (ADAM) TODO cite Diederik P. Kingma and Jimmy Ba. Adam: A Method for Stochastic Optimization. arXiv, 2011.

\section{Ensemble Models}
It is well established in psychology, economics, and other fields that although an individual may be inaccurate in their predictions, the average of many individuals is often more accurate than any one individual, a term referred to as the "wisdom of the crowd" TODO cite a bunch of people here.
This principle can be applied to machine learning, by creating an ensemble of models, where ntraining data is split between multiple models, and the predictions of each model are averaged to create a final prediction TODO cite Gavin Brown. Ensemble learning. In Encyclopedia of Machine Learning, pages 312 320. Springer, 2011. 
